\section{提案手法の概要}

本研究で提案するNekoNetは、猫画像認識に特化した深層学習システムである。本システムは、以下の3つのコンポーネントから構成される。

\subsection{システム全体像}

\begin{enumerate}
    \item \textbf{Cat Detector}:入力画像から猫の存在を検出し、バウンディングボックスを出力するモジュール。YOLOv8\cite{YOLOv8}をベースに、猫特化型の改良を加えている。
    
    \item \textbf{Breed Classifier}:検出された猫領域から品種を分類するモジュール。マルチスケール特徴抽出機構により、細粒度な分類を実現する。
    
    \item \textbf{Individual Identifier}:同一品種内の個体を識別するモジュール。メトリック学習に基づき、個体ごとの特徴ベクトルを生成する。
\end{enumerate}

\subsection{ミッションステイトメント}

予備調査の結果を踏まえ、本システムのミッションステイトメントを以下のように定義した。

\begin{quote}
「NekoNetは、一般ユーザーがスマートフォンで撮影した猫画像から、リアルタイムで猫の検出・品種分類・個体識別を行い、ペットライフの質向上に貢献するシステムである」
\end{quote}

\subsection{ミッション要求}

ミッションステイトメントに基づき、以下の4つのミッション要求を設定した。

\begin{description}
    \item[MR1] システムは、画像中の猫を90\%以上の精度(mAP@0.5)で検出できること
    \item[MR2] システムは、50品種以上の猫を80\%以上の精度で分類できること
    \item[MR3] システムは、登録済み個体を90\%以上の精度で識別できること
    \item[MR4] システムは、スマートフォン上で30FPS以上の速度で動作すること
\end{description}
