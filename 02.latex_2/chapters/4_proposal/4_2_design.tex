\section{手法の設計}

本節では、システムズエンジニアリング\cite{INCOSE2015}に基づく設計プロセスを説明する。

\subsection{ステイクホルダー分析}

本システムのステイクホルダーを以下のように特定した。

\begin{itemize}
    \item \textbf{主要ステイクホルダー}:猫飼育者(一般ユーザー)
    \item \textbf{二次ステイクホルダー}:動物病院、ペットショップ、ペット保険会社
    \item \textbf{開発ステイクホルダー}:システム開発者、データサイエンティスト
    \item \textbf{規制ステイクホルダー}:個人情報保護、動物愛護に関する規制当局
\end{itemize}

\subsection{ステイクホルダー要求}

各ステイクホルダーのニーズを分析し、以下のステイクホルダー要求を導出した。

\begin{description}
    \item[StR1] システムは、直感的な操作で利用できること(猫飼育者)
    \item[StR2] システムは、オフライン環境でも基本機能が利用できること(猫飼育者)
    \item[StR3] システムは、診療記録との連携が可能であること(動物病院)
    \item[StR4] システムは、個人情報を適切に保護すること(規制当局)
    \item[StR5] システムは、モデルの更新・改善が容易であること(開発者)
\end{description}

\subsection{システム要求}

ミッション要求とステイクホルダー要求を統合し、以下のシステム要求を定義した。

\begin{description}
    \item[SyR1] 検出モジュールは、mAP@0.5で92\%以上の精度を達成すること
    \item[SyR2] 分類モジュールは、Top-1 Accuracyで85\%以上を達成すること
    \item[SyR3] 識別モジュールは、Rank-1 Accuracyで92\%以上を達成すること
    \item[SyR4] 全体の処理時間は、1フレームあたり33ms以下であること(30FPS相当)
    \item[SyR5] モデルサイズは、合計10MB以下であること
    \item[SyR6] オンデバイス推論をサポートすること
\end{description}
