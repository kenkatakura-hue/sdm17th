\section{実験方法}

\subsection{データセット}

本実験では、独自に構築した「Japanese Domestic Cat Dataset(JDCD)」を使用した。データセットの概要は以下の通りである。

\begin{itemize}
    \item \textbf{総画像数}:15,000枚
    \item \textbf{品種数}:50品種(日本で人気の品種を中心に選定)
    \item \textbf{収集方法}:SNSからの収集(許諾取得済み)、協力家庭からの提供
    \item \textbf{アノテーション}:バウンディングボックス、品種ラベル、個体ID
    \item \textbf{分割}:訓練セット(10,500枚)、検証セット(1,500枚)、テストセット(3,000枚)
\end{itemize}

個体識別の評価には、サブセットとして100個体(各個体20枚以上)のデータを使用した。

\subsection{比較手法}

以下の既存手法との比較を行った。

\begin{itemize}
    \item YOLOv8-nano\cite{YOLOv8}:汎用物体検出モデルのベースライン
    \item YOLOv8-small\cite{YOLOv8}:中規模モデル
    \item EfficientDet-D0:効率的なアーキテクチャを持つ検出モデル
    \item ResNet-50\cite{He2016} + Fine-tuning:品種分類のベースライン
\end{itemize}

\subsection{評価指標}

\begin{itemize}
    \item \textbf{検出}:mAP@0.5, mAP@0.5:0.95, Precision, Recall
    \item \textbf{分類}:Top-1 Accuracy, Top-5 Accuracy, F1-Score
    \item \textbf{識別}:Rank-1 Accuracy, Rank-5 Accuracy, mAP
    \item \textbf{実行性能}:FPS(GPU/CPU/Mobile)、モデルサイズ(MB)、メモリ使用量
\end{itemize}

\subsection{実験環境}

\begin{itemize}
    \item \textbf{学習}:NVIDIA RTX 4090、PyTorch 2.0、CUDA 12.0
    \item \textbf{GPU推論}:NVIDIA RTX 4090
    \item \textbf{CPU推論}:Intel Core i9-13900K
    \item \textbf{モバイル推論}:iPhone 14 Pro(A16 Bionic)、Pixel 7(Google Tensor G2)
\end{itemize}
