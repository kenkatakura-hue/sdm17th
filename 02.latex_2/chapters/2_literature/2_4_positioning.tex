\section{先行研究の課題と本研究の位置づけ}

先行研究のレビューから、以下の3つの課題が明らかになった。

\subsection{課題1:猫特化型の統合システムの不在}

既存研究は、検出・分類・個体識別といった個別タスクに焦点を当てており、これらを統合的に扱うシステムの研究は限定的である。実用的なアプリケーションでは、これらのタスクをシームレスに連携させる必要がある。

\subsection{課題2:日本市場向けデータセットの不足}

既存のベンチマークデータセット\cite{Parkhi2012}は欧米の猫画像が中心であり、日本で人気の品種や日本の家庭環境での撮影条件を反映したデータが不足している。これにより、国内での実用化において精度低下のリスクがある。

\subsection{課題3:精度と軽量性の両立}

高精度なモデルは計算量が大きく、エッジデバイスでの実行が困難である。一方、軽量モデルは精度が低下する傾向にある。猫画像認識において、この両立を実現した研究は少ない。

\subsection{本研究の位置づけ}

本研究は、上記の課題を解決するため、以下の特徴を持つシステムを提案する。

\begin{enumerate}
    \item 検出から個体識別までを統合したエンドツーエンドシステム
    \item 日本の家庭猫データセットによる学習・評価
    \item 軽量アーキテクチャと知識蒸留の組み合わせによる精度・軽量性の両立
\end{enumerate}

さらに、システムズエンジニアリングのアプローチ\cite{INCOSE2015}を適用し、体系的な開発プロセスを実践することで、AIシステム開発の方法論にも貢献する。
