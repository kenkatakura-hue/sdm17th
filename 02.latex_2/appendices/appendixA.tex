\chapter{データセット詳細}

\section{品種一覧}

Japanese Domestic Cat Dataset(JDCD)に含まれる50品種を以下に示す。

\begin{enumerate}
    \item アビシニアン
    \item アメリカンカール
    \item アメリカンショートヘア
    \item エキゾチックショートヘア
    \item オシキャット
    \item キジトラ(混血)
    \item サイベリアン
    \item サバトラ(混血)
    \item シャム
    \item シャルトリュー
    \item シンガプーラ
    \item スコティッシュフォールド
    \item スフィンクス
    \item セルカークレックス
    \item ソマリ
    \item ターキッシュアンゴラ
    \item チンチラ
    \item トンキニーズ
    \item ノルウェージャンフォレストキャット
    \item ハバナブラウン
    \item バーマン
    \item バーミーズ
    \item ヒマラヤン
    \item ブリティッシュショートヘア
    \item ベンガル
    \item ペルシャ
    \item ボンベイ
    \item マンクス
    \item マンチカン
    \item ミヌエット
    \item メインクーン
    \item ラガマフィン
    \item ラグドール
    \item ロシアンブルー
    \item 三毛猫(混血)
    \item 白猫(混血)
    \item 黒猫(混血)
    \item 茶トラ(混血)
    \item その他の品種(12種)
\end{enumerate}

\section{アノテーション仕様}

各画像に対して、以下のアノテーションを付与した。

\begin{itemize}
    \item \textbf{バウンディングボックス}:猫の全身を囲む矩形(COCO形式)
    \item \textbf{品種ラベル}:上記50クラスのいずれか
    \item \textbf{個体ID}:同一個体を追跡するための一意識別子(サブセットのみ)
    \item \textbf{撮影環境}:屋内/屋外、照明条件
    \item \textbf{姿勢}:立ち/座り/寝そべり/丸まり/その他
\end{itemize}
