% 慶應義塾大学大学院 SDM研究科 修士論文テンプレート
% Compiler: XeLaTeX

\documentclass[12pt,a4paper,openany]{bxjsreport}
\usepackage{xeCJK}
\setCJKmainfont{IPAexMincho}
\setCJKsansfont{IPAexGothic}

% パッケージ
\usepackage{graphicx}
\usepackage{amsmath,amssymb}
\usepackage{booktabs}
\usepackage{longtable}
\usepackage{tabularx}
\usepackage{hyperref}
\usepackage{geometry}
\usepackage{titlesec}
\geometry{top=25mm, bottom=25mm, left=25mm, right=25mm}

% ハイパーリンク設定 + PDFプロパティ(KOARA提出要件)
\hypersetup{
    pdftitle={深層学習を用いた猫画像認識システムの設計と評価},
    pdfauthor={慶應義塾大学大学院システムデザイン・マネジメント研究科},
    colorlinks=true,
    linkcolor=black,
    citecolor=black,
    urlcolor=blue,
    linktoc=all
}

% 図表キャプション用(日本語対応)
\renewcommand{\figurename}{図}
\renewcommand{\tablename}{表}
\renewcommand{\listfigurename}{図目次}
\renewcommand{\listtablename}{表目次}

% ========== 見出しスタイル設定 ==========
% 見出し1: 章(第1章 タイトル)
\titleformat{\chapter}[block]
  {\normalfont\Large\bfseries}
  {第\thechapter 章}{1em}{}
\titlespacing*{\chapter}{0pt}{0pt}{20pt}

% 見出し2: 節(1.1 タイトル)
\titleformat{\section}
  {\normalfont\large\bfseries}
  {\thesection}{1em}{}
\titlespacing*{\section}{0pt}{20pt}{10pt}

% 見出し3: 項(1.1.1 タイトル)
\titleformat{\subsection}
  {\normalfont\normalsize\bfseries}
  {\thesubsection}{1em}{}
\titlespacing*{\subsection}{0pt}{15pt}{8pt}

% 見出し4: 小項((1) タイトル)
\renewcommand{\thesubsubsection}{(\arabic{subsubsection})}
\titleformat{\subsubsection}
  {\normalfont\normalsize\bfseries}
  {\thesubsubsection}{1em}{}
\titlespacing*{\subsubsection}{0pt}{12pt}{6pt}

\begin{document}

% ========== 前付け ==========
\pagenumbering{roman}

% 表紙
\begin{center}
\vspace*{2cm}

{\LARGE 博士論文}

\vspace{1.5cm}

{\huge\bfseries 猫画像認識AIの効率化手法に関する研究}

\vspace{0.3cm}

{\Large --- NekoNet アーキテクチャの提案と評価 ---}

\vspace{2.5cm}

{\Large 慶應義塾大学大学院}

{\Large システムデザイン・マネジメント研究科}

\vspace{1cm}

{\Large 慶應 太郎}

\vspace{1cm}

{\large 指導教員:福沢 諭吉}

\vspace{1.5cm}

{\Large 2025年3月}

\end{center}

% 日本語要旨
% abstract.tex
% 論文要旨(SDM形式)

\thispagestyle{plain}

\begin{center}
{\large\textbf{論 文 要 旨}}
\end{center}

\vspace{1em}

\noindent
\begin{tabular}{|p{2cm}|p{3.2cm}|p{1.3cm}|p{5.7cm}|}
\hline
学籍番号 & 12345678 & 氏 名 & 山田太郎 \\
\hline
論文題目: & \multicolumn{3}{p{10.6cm}|}{猫画像認識AIの効率化手法に関する研究 --- NekoNetアーキテクチャの提案と評価 ---} \\
\hline
\multicolumn{4}{|p{13.1cm}|}{
\vspace{0.3em}
\textbf{(内容の要旨)}

\hspace{1em}深層学習を用いた画像認識技術は急速に発展しているが、猫画像の認識においては毛並みパターンの多様性や姿勢変化への対応が課題となっている。既存の汎用画像認識モデルでは、これらの猫特有の課題に対して十分な対応ができておらず、猫の個体識別においては80\%程度の精度に留まることが報告されている。

\hspace{1em}本研究の目的は、(1) 猫画像認識に関連する既存手法を体系的にレビューし、現状の課題と研究ギャップを明らかにすること、(2) 猫特有の形態学的特徴を活用した新規ニューラルネットワークアーキテクチャ「NekoNet」をISO 15288に基づくシステムズエンジニアリングアプローチを用いて設計・実装・評価すること、の2点である。

\hspace{1em}体系的文献レビュー(SLR)により、417件の論文から猫認識に関連する42手法を抽出・分類した。分析の結果、既存手法の大半が猫特有の形態学的特徴を活用しておらず、精度と速度のトレードオフが大きいことを発見した。この課題に対応するため、猫の耳・目・尾の形態学的特徴に着目したNeko Attention Moduleを設計し、MobileNetV3をベースとした軽量アーキテクチャNekoNetを実装した。評価実験の結果、提案手法はmAP@0.5で92.6\%の認識精度と48FPSの処理速度を達成し、従来手法(YOLOv8-nano、MobileNet-SSD、EfficientDet-D0)を有意に上回る性能を示した。

\hspace{1em}本研究は、猫画像認識における形態学的特徴の体系化と、それを活用した効率的な認識手法の確立に貢献する。
\vspace{0.3em}
} \\
\hline
\multicolumn{4}{|p{13.1cm}|}{
\textbf{キーワード(5語)}

深層学習、画像認識、畳み込みニューラルネットワーク、アテンション機構、NekoNet
\vspace{0.2em}
} \\
\hline
\end{tabular}
\clearpage

% 英語要旨
% abstract-en.tex
% Abstract(SDM形式)

\thispagestyle{plain}

\begin{center}
{\large\textbf{Abstract}}
\end{center}

\vspace{1em}

\noindent
\begin{tabular}{|p{2.2cm}|p{3cm}|p{1.3cm}|p{5.7cm}|}
\hline
Student ID & 12345678 & Name & Taro Yamada \\
\hline
Thesis Title: & \multicolumn{3}{p{10.4cm}|}{Design and Evaluation of a Deep Learning-based Cat Image Recognition System} \\
\hline
\multicolumn{4}{|p{13.1cm}|}{
\vspace{0.3em}
\textbf{(Summary)}

\hspace{1em}With the expansion of the pet industry, demand for automatic cat image recognition technology has been growing. However, existing general-purpose object detection models inadequately address cat-specific morphological features, particularly showing accuracy limitations in individual identification within multi-cat environments and fine-grained breed classification.

\hspace{1em}This study aims to: (1) design and implement ``NekoNet,'' a deep learning system specialized for cat image recognition, achieving detection and classification accuracy surpassing existing methods; (2) realize a lightweight model enabling real-time inference on edge devices such as smartphones; and (3) demonstrate the effectiveness of systematic AI system development applying systems engineering processes.

\hspace{1em}The proposed NekoNet system features a multi-scale feature extraction mechanism that integrates analysis of cat faces, body types, and coat patterns; an attention mechanism robust to occlusion and pose variations; and a lightweight architecture enabling execution on edge devices. Evaluation experiments using a proprietary Japanese domestic cat dataset (15,000 images, 50 breeds) showed that the proposed method achieved mAP@0.5 = 92.6\% in object detection, surpassing existing method YOLOv8 (87.3\%) by 5.3 percentage points. Breed classification achieved Top-1 Accuracy of 87.3\%, and individual identification achieved Rank-1 Accuracy of 93.4\%. Furthermore, through a combination of knowledge distillation, quantization, and pruning, a model size of 7.8MB and processing speed of 48 FPS in mobile environments were realized.

\hspace{1em}This study contributes to establishing a new benchmark for cat image recognition, presenting design guidelines for practical lightweight models, and demonstrating an AI system development process using systems engineering.
\vspace{0.3em}
} \\
\hline
\multicolumn{4}{|p{13.1cm}|}{
\textbf{Keywords (5 words)}

Deep Learning, Object Detection, Cat Image Recognition, Model Compression, Systems Engineering
\vspace{0.2em}
} \\
\hline
\end{tabular}
\clearpage

% 目次
\tableofcontents
\clearpage

% 図目次
\listoffigures
\clearpage

% 表目次
\listoftables
\clearpage

% ========== 本文 ==========
\pagenumbering{arabic}

% 第1章 序論
\chapter{序論}
\input{chapters/1_intro/1_1_background}
\input{chapters/1_intro/1_2_objective}
\input{chapters/1_intro/1_3_novelty}
\section{本論文の構成}

本論文は全7章から構成される。

\textbf{第1章}では、本研究の背景、目的、新規性、および論文の構成を述べた。

\textbf{第2章}では、先行研究のレビューを行う。深層学習に基づく物体検出手法、猫画像認識の既存研究、および軽量化技術について整理し、本研究の位置づけを明確にする。

\textbf{第3章}では、予備調査の結果を報告する。既存システムの分析、ユーザーインタビュー、およびプロトタイプテストを通じて得られた知見を整理し、システム設計への示唆を導出する。

\textbf{第4章}では、提案手法であるNekoNetの設計を詳述する。システムズエンジニアリングに基づく要求定義、アーキテクチャ設計、およびアルゴリズムの詳細を説明する。

\textbf{第5章}では、評価実験について述べる。実験の目的、方法、結果を報告し、提案手法の有効性を検証する。

\textbf{第6章}では、実験結果に基づく考察を行う。得られた知見の解釈、および本研究の限界について議論する。

\textbf{第7章}では、本研究の結論と今後の展望を述べる。


% 第2章 先行研究
\chapter{先行研究}
\section{深層学習に基づく物体検出}

物体検出は、画像中の物体の位置(バウンディングボックス)とクラスを同時に推定するタスクである。深層学習の発展以前は、HOG特徴量とSVMを組み合わせた手法が主流であったが、2012年のAlexNetの登場以降、畳み込みニューラルネットワーク(CNN)に基づく手法が急速に発展した。

物体検出の深層学習手法は、大きく2段階検出器と1段階検出器に分類される。2段階検出器の代表例であるR-CNNファミリーは、領域提案ネットワーク(RPN)で候補領域を生成した後、各領域を分類する。Fast R-CNN、Faster R-CNNと改良が重ねられ、高精度な検出が可能となった。一方、YOLOやSSDに代表される1段階検出器は、領域提案を省略して直接検出を行うため、リアルタイム処理に適している。

近年では、Transformerアーキテクチャを物体検出に応用したDETR(DEtection TRansformer)が登場し、エンドツーエンドの学習を実現している。また、YOLOv8やRT-DETRなど、精度と速度のバランスに優れた最新手法も提案されている。

しかしながら、これらの汎用手法は特定のドメイン(本研究では猫画像)に最適化されておらず、ドメイン固有の特徴を活かした手法の開発が求められている。

\section{猫画像認識に関する先行研究}

猫画像認識に関する研究は、主に以下の3つのタスクに分類される。

\subsection{猫検出}

画像中から猫の存在を検出するタスクである。Parkhi et al. (2012) は Oxford-IIIT Pet Datasetを公開し、37品種の犬猫画像を用いたベンチマークを確立した。Zhang et al. (2008) は、猫の顔検出に特化したカスケード分類器を提案し、毛色や模様のバリエーションに対応した。近年では、汎用物体検出モデルの転移学習により、高い検出精度が達成されている。

\subsection{猫種分類}

検出された猫を品種ごとに分類するタスクである。細粒度画像認識(Fine-Grained Visual Categorization)の一種として研究されており、品種間の微細な差異を捉えることが課題となる。Liu et al. (2012) は、局所特徴量と大域特徴量を組み合わせた手法を提案した。また、アテンション機構を用いて識別に重要な領域を自動的に特定する研究も行われている。

\subsection{個体識別}

同一品種内の個体を識別するタスクである。ペットの迷子捜索や健康管理において重要なアプリケーションである。顔認証技術を応用したCat Face Recognitionや、毛色パターンを用いた手法が提案されている。Kumar et al. (2020) は、Siamese Networkを用いた猫の個体識別システムを開発し、95\%以上の識別精度を報告している。

これらの先行研究は個別のタスクに焦点を当てているが、検出から個体識別までを統合的に扱うシステムの研究は限定的である。

\input{chapters/2_literature/2_3_domain_c}
\section{先行研究の課題と本研究の位置づけ}

先行研究のレビューから、以下の3つの課題が明らかになった。

\subsection{課題1:猫特化型の統合システムの不在}

既存研究は、検出・分類・個体識別といった個別タスクに焦点を当てており、これらを統合的に扱うシステムの研究は限定的である。実用的なアプリケーションでは、これらのタスクをシームレスに連携させる必要がある。

\subsection{課題2:日本市場向けデータセットの不足}

既存のベンチマークデータセット\cite{Parkhi2012}は欧米の猫画像が中心であり、日本で人気の品種や日本の家庭環境での撮影条件を反映したデータが不足している。これにより、国内での実用化において精度低下のリスクがある。

\subsection{課題3:精度と軽量性の両立}

高精度なモデルは計算量が大きく、エッジデバイスでの実行が困難である。一方、軽量モデルは精度が低下する傾向にある。猫画像認識において、この両立を実現した研究は少ない。

\subsection{本研究の位置づけ}

本研究は、上記の課題を解決するため、以下の特徴を持つシステムを提案する。

\begin{enumerate}
    \item 検出から個体識別までを統合したエンドツーエンドシステム
    \item 日本の家庭猫データセットによる学習・評価
    \item 軽量アーキテクチャと知識蒸留の組み合わせによる精度・軽量性の両立
\end{enumerate}

さらに、システムズエンジニアリングのアプローチ\cite{INCOSE2015}を適用し、体系的な開発プロセスを実践することで、AIシステム開発の方法論にも貢献する。

\input{chapters/2_literature/2_5_summary}

% 第3章 予備調査
\chapter{予備調査}
\input{chapters/3_preliminary/3_1_purpose}
\input{chapters/3_preliminary/3_2_method}
\section{調査結果}

\subsection{既存システム分析の結果}

10アプリケーションの分析から、以下の知見が得られた。

\textbf{機能面}:品種識別機能を持つアプリは7件、個体識別機能を持つアプリは2件であった。複数頭の同時検出に対応しているアプリは4件にとどまった。

\textbf{精度面}:自前の100枚テストでは、最高精度のアプリでも品種識別の正解率は73\%であった。特に、混血猫や希少品種での誤認識が多く見られた。また、暗い環境や猫が動いている場面での認識精度が著しく低下した。

\textbf{ユーザビリティ}:ユーザーレビューの分析から、「認識に時間がかかる」「オフラインで使えない」「日本の猫種に対応していない」といった不満が頻出していた。

\subsection{ユーザーインタビューの結果}

インタビューの質的分析から、以下の主要なニーズが抽出された。

\begin{enumerate}
    \item \textbf{迷子対策}:猫が迷子になった際に、写真から個体を特定できる機能への要望(飼育者12名が言及)
    \item \textbf{健康管理}:体型や毛並みの変化を追跡し、健康状態を把握したいというニーズ(飼育者8名、獣医師3名が言及)
    \item \textbf{多頭識別}:複数の猫を飼育している場合に、個体ごとの記録を管理したいというニーズ(飼育者6名が言及)
    \item \textbf{リアルタイム性}:撮影と同時に結果が得られることへの期待(全グループで言及)
\end{enumerate}

\subsection{プロトタイプテストの結果}

YOLOv8ベースのプロトタイプは、500枚のテスト画像に対して以下の結果を示した。

\begin{itemize}
    \item 検出精度(mAP@0.5):78.3\%
    \item 品種分類精度(Top-1 Accuracy):65.2\%
    \item 処理速度:42 FPS(GPU環境)、8 FPS(CPU環境)
\end{itemize}

失敗パターンの分析から、以下の課題が明らかになった:(1)類似品種間の混同、(2)部分的に隠れた猫の検出失敗、(3)丸まった姿勢での認識精度低下。

\input{chapters/3_preliminary/3_4_summary}

% 第4章 提案手法
\chapter{提案手法}
\section{提案手法の概要}

本研究で提案するNekoNetは、猫画像認識に特化した深層学習システムである。本システムは、以下の3つのコンポーネントから構成される。

\subsection{システム全体像}

\begin{enumerate}
    \item \textbf{Cat Detector}:入力画像から猫の存在を検出し、バウンディングボックスを出力するモジュール。YOLOv8\cite{YOLOv8}をベースに、猫特化型の改良を加えている。
    
    \item \textbf{Breed Classifier}:検出された猫領域から品種を分類するモジュール。マルチスケール特徴抽出機構により、細粒度な分類を実現する。
    
    \item \textbf{Individual Identifier}:同一品種内の個体を識別するモジュール。メトリック学習に基づき、個体ごとの特徴ベクトルを生成する。
\end{enumerate}

\subsection{ミッションステイトメント}

予備調査の結果を踏まえ、本システムのミッションステイトメントを以下のように定義した。

\begin{quote}
「NekoNetは、一般ユーザーがスマートフォンで撮影した猫画像から、リアルタイムで猫の検出・品種分類・個体識別を行い、ペットライフの質向上に貢献するシステムである」
\end{quote}

\subsection{ミッション要求}

ミッションステイトメントに基づき、以下の4つのミッション要求を設定した。

\begin{description}
    \item[MR1] システムは、画像中の猫を90\%以上の精度(mAP@0.5)で検出できること
    \item[MR2] システムは、50品種以上の猫を80\%以上の精度で分類できること
    \item[MR3] システムは、登録済み個体を90\%以上の精度で識別できること
    \item[MR4] システムは、スマートフォン上で30FPS以上の速度で動作すること
\end{description}

\section{手法の設計}

本節では、システムズエンジニアリング\cite{INCOSE2015}に基づく設計プロセスを説明する。

\subsection{ステイクホルダー分析}

本システムのステイクホルダーを以下のように特定した。

\begin{itemize}
    \item \textbf{主要ステイクホルダー}:猫飼育者(一般ユーザー)
    \item \textbf{二次ステイクホルダー}:動物病院、ペットショップ、ペット保険会社
    \item \textbf{開発ステイクホルダー}:システム開発者、データサイエンティスト
    \item \textbf{規制ステイクホルダー}:個人情報保護、動物愛護に関する規制当局
\end{itemize}

\subsection{ステイクホルダー要求}

各ステイクホルダーのニーズを分析し、以下のステイクホルダー要求を導出した。

\begin{description}
    \item[StR1] システムは、直感的な操作で利用できること(猫飼育者)
    \item[StR2] システムは、オフライン環境でも基本機能が利用できること(猫飼育者)
    \item[StR3] システムは、診療記録との連携が可能であること(動物病院)
    \item[StR4] システムは、個人情報を適切に保護すること(規制当局)
    \item[StR5] システムは、モデルの更新・改善が容易であること(開発者)
\end{description}

\subsection{システム要求}

ミッション要求とステイクホルダー要求を統合し、以下のシステム要求を定義した。

\begin{description}
    \item[SyR1] 検出モジュールは、mAP@0.5で92\%以上の精度を達成すること
    \item[SyR2] 分類モジュールは、Top-1 Accuracyで85\%以上を達成すること
    \item[SyR3] 識別モジュールは、Rank-1 Accuracyで92\%以上を達成すること
    \item[SyR4] 全体の処理時間は、1フレームあたり33ms以下であること(30FPS相当)
    \item[SyR5] モデルサイズは、合計10MB以下であること
    \item[SyR6] オンデバイス推論をサポートすること
\end{description}

\section{手法の詳細}

本節では、NekoNetの各コンポーネントの技術的詳細を説明する。

\subsection{Cat Detector}

Cat Detectorは、YOLOv8-nanoをベースに以下の改良を加えた検出モジュールである。

\textbf{バックボーン}:CSPDarknetをベースに、猫の特徴抽出に適した改良を加えた。具体的には、毛色パターンの検出に有効な低レベル特徴を保持するスキップ接続を追加した。

\textbf{ネック}:PANet(Path Aggregation Network)を採用し、マルチスケールの特徴マップを統合する。これにより、大きさの異なる猫(子猫から成猫まで)を同一モデルで検出可能とした。

\textbf{ヘッド}:アンカーフリーの検出ヘッドを採用し、猫の多様な姿勢に対応する。

\subsection{Breed Classifier}

Breed Classifierは、細粒度画像認識のための分類モジュールである。

\textbf{マルチスケール特徴抽出}:猫の顔(256×256)、上半身(384×384)、全身(512×512)の3つのスケールで特徴を抽出し、統合する。これにより、顔の特徴(耳の形、目の色)と体型の特徴(毛並み、体格)を相補的に活用できる。

\textbf{アテンション機構}:CBAM(Convolutional Block Attention Module)を導入し、品種識別に重要な領域に注目させる。これにより、背景ノイズの影響を軽減する。

\textbf{分類ヘッド}:50クラス(50品種)の分類を行う全結合層。「混血」クラスも含め、実用的な分類体系を構築した。

\subsection{Individual Identifier}

Individual Identifierは、メトリック学習に基づく個体識別モジュールである。

\textbf{特徴抽出}:ResNet-18ベースのエンコーダにより、128次元の特徴ベクトルを生成する。

\textbf{損失関数}:Triplet Lossを採用し、同一個体の特徴ベクトルが近く、異なる個体の特徴ベクトルが遠くなるよう学習する。

\textbf{識別方法}:登録済み個体の特徴ベクトルとのコサイン類似度を計算し、閾値以上であれば同一個体と判定する。

\subsection{軽量化手法}

モデルサイズと推論速度の要件を満たすため、以下の軽量化手法を適用した。

\begin{enumerate}
    \item \textbf{知識蒸留}:大規模教師モデル(YOLOv8-large)から小規模生徒モデルへ知識を転移
    \item \textbf{INT8量子化}:推論時の演算を8ビット整数で実行
    \item \textbf{チャネル枝刈り}:重要度の低いチャネルを削除し、モデルを圧縮
\end{enumerate}

\input{chapters/4_proposal/4_4_summary}

% 第5章 評価実験
\chapter{評価実験}
\section{実験目的}

評価実験の目的は、提案手法であるNekoNetが、第4章で定義したシステム要求を満たしているかを検証することである。

具体的には、以下の4点を検証する。

\begin{enumerate}
    \item \textbf{検出性能の検証}:Cat Detectorが、mAP@0.5で92\%以上の精度を達成しているか(SyR1)
    \item \textbf{分類性能の検証}:Breed Classifierが、Top-1 Accuracyで85\%以上を達成しているか(SyR2)
    \item \textbf{識別性能の検証}:Individual Identifierが、Rank-1 Accuracyで92\%以上を達成しているか(SyR3)
    \item \textbf{実行性能の検証}:処理速度が30FPS以上(33ms/frame以下)、モデルサイズが10MB以下であるか(SyR4, SyR5)
\end{enumerate}

また、既存手法との比較を通じて、提案手法の優位性を示す。

\section{実験方法}

\subsection{データセット}

本実験では、独自に構築した「Japanese Domestic Cat Dataset(JDCD)」を使用した。データセットの概要は以下の通りである。

\begin{itemize}
    \item \textbf{総画像数}:15,000枚
    \item \textbf{品種数}:50品種(日本で人気の品種を中心に選定)
    \item \textbf{収集方法}:SNSからの収集(許諾取得済み)、協力家庭からの提供
    \item \textbf{アノテーション}:バウンディングボックス、品種ラベル、個体ID
    \item \textbf{分割}:訓練セット(10,500枚)、検証セット(1,500枚)、テストセット(3,000枚)
\end{itemize}

個体識別の評価には、サブセットとして100個体(各個体20枚以上)のデータを使用した。

\subsection{比較手法}

以下の既存手法との比較を行った。

\begin{itemize}
    \item YOLOv8-nano\cite{YOLOv8}:汎用物体検出モデルのベースライン
    \item YOLOv8-small\cite{YOLOv8}:中規模モデル
    \item EfficientDet-D0:効率的なアーキテクチャを持つ検出モデル
    \item ResNet-50\cite{He2016} + Fine-tuning:品種分類のベースライン
\end{itemize}

\subsection{評価指標}

\begin{itemize}
    \item \textbf{検出}:mAP@0.5, mAP@0.5:0.95, Precision, Recall
    \item \textbf{分類}:Top-1 Accuracy, Top-5 Accuracy, F1-Score
    \item \textbf{識別}:Rank-1 Accuracy, Rank-5 Accuracy, mAP
    \item \textbf{実行性能}:FPS(GPU/CPU/Mobile)、モデルサイズ(MB)、メモリ使用量
\end{itemize}

\subsection{実験環境}

\begin{itemize}
    \item \textbf{学習}:NVIDIA RTX 4090、PyTorch 2.0、CUDA 12.0
    \item \textbf{GPU推論}:NVIDIA RTX 4090
    \item \textbf{CPU推論}:Intel Core i9-13900K
    \item \textbf{モバイル推論}:iPhone 14 Pro(A16 Bionic)、Pixel 7(Google Tensor G2)
\end{itemize}

\section{実験結果}

\subsection{検出性能}

Cat Detectorの検出性能を表\ref{tab:detection}に示す。

\begin{table}[htbp]
\centering
\caption{検出性能の比較}
\label{tab:detection}
\begin{tabular}{lcccc}
\toprule
手法 & mAP@0.5 & mAP@0.5:0.95 & Precision & Recall \\
\midrule
YOLOv8-nano\cite{YOLOv8} & 87.3\% & 62.1\% & 89.2\% & 84.5\% \\
YOLOv8-small\cite{YOLOv8} & 90.1\% & 67.8\% & 91.5\% & 87.3\% \\
EfficientDet-D0 & 85.6\% & 59.4\% & 87.8\% & 82.1\% \\
\textbf{NekoNet (Ours)} & \textbf{92.6\%} & \textbf{71.2\%} & \textbf{93.8\%} & \textbf{90.2\%} \\
\bottomrule
\end{tabular}
\end{table}

提案手法は、mAP@0.5で92.6\%を達成し、システム要求(SyR1: 92\%以上)を満たした。また、すべての比較手法を上回る性能を示した。

\subsection{分類性能}

Breed Classifierの分類性能を表\ref{tab:classification}に示す。

\begin{table}[htbp]
\centering
\caption{分類性能の比較}
\label{tab:classification}
\begin{tabular}{lccc}
\toprule
手法 & Top-1 Acc. & Top-5 Acc. & F1-Score \\
\midrule
ResNet-50\cite{He2016} & 78.2\% & 94.1\% & 0.772 \\
EfficientNet-B0\cite{Tan2019} & 81.5\% & 95.3\% & 0.808 \\
\textbf{NekoNet (Ours)} & \textbf{87.3\%} & \textbf{97.8\%} & \textbf{0.865} \\
\bottomrule
\end{tabular}
\end{table}

提案手法は、Top-1 Accuracyで87.3\%を達成し、システム要求(SyR2: 85\%以上)を満たした。

\subsection{識別性能}

Individual Identifierの識別性能を表\ref{tab:identification}に示す。

\begin{table}[htbp]
\centering
\caption{個体識別性能}
\label{tab:identification}
\begin{tabular}{lccc}
\toprule
手法 & Rank-1 Acc. & Rank-5 Acc. & mAP \\
\midrule
Siamese Network & 88.5\% & 96.2\% & 0.872 \\
ArcFace & 91.2\% & 97.8\% & 0.903 \\
\textbf{NekoNet (Ours)} & \textbf{93.4\%} & \textbf{98.5\%} & \textbf{0.921} \\
\bottomrule
\end{tabular}
\end{table}

提案手法は、Rank-1 Accuracyで93.4\%を達成し、システム要求(SyR3: 92\%以上)を満たした。

\subsection{実行性能}

実行性能を表\ref{tab:performance}に示す。

\begin{table}[htbp]
\centering
\caption{実行性能の比較}
\label{tab:performance}
\begin{tabular}{lcccc}
\toprule
手法 & サイズ(MB) & GPU(FPS) & CPU(FPS) & Mobile(FPS) \\
\midrule
YOLOv8-nano\cite{YOLOv8} & 6.2 & 312 & 45 & 38 \\
YOLOv8-small\cite{YOLOv8} & 21.5 & 186 & 28 & 22 \\
\textbf{NekoNet (Ours)} & \textbf{7.8} & \textbf{285} & \textbf{52} & \textbf{48} \\
\bottomrule
\end{tabular}
\end{table}

提案手法は、モデルサイズ7.8MB(SyR5: 10MB以下)、モバイル環境で48FPS(SyR4: 30FPS以上)を達成し、すべての実行性能要求を満たした。

\section{検証と妥当性確認}

\subsection{システム要求の検証結果}

表\ref{tab:verification}に、各システム要求の検証結果をまとめる。

\begin{table}[htbp]
\centering
\caption{システム要求の検証結果}
\label{tab:verification}
\begin{tabular}{llcc}
\toprule
ID & 要求内容 & 目標値 & 達成値 \\
\midrule
SyR1 & 検出精度(mAP@0.5) & $\geq$ 92\% & 92.6\% ✓ \\
SyR2 & 分類精度(Top-1 Acc.) & $\geq$ 85\% & 87.3\% ✓ \\
SyR3 & 識別精度(Rank-1 Acc.) & $\geq$ 92\% & 93.4\% ✓ \\
SyR4 & 処理速度(Mobile) & $\geq$ 30 FPS & 48 FPS ✓ \\
SyR5 & モデルサイズ & $\leq$ 10 MB & 7.8 MB ✓ \\
SyR6 & オンデバイス推論 & 対応 & 対応 ✓ \\
\bottomrule
\end{tabular}
\end{table}

すべてのシステム要求が達成されたことを確認した。

\subsection{妥当性確認}

システムが実際のユーザーニーズを満たしているかを確認するため、以下の妥当性確認を実施した。

\textbf{ユーザーテスト}:猫飼育者10名に対し、プロトタイプアプリケーションを2週間使用してもらい、満足度調査を実施した。5段階評価(5が最高)で、以下の結果を得た。

\begin{itemize}
    \item 認識精度への満足度:4.3
    \item 処理速度への満足度:4.6
    \item 使いやすさへの満足度:4.1
    \item 総合満足度:4.2
\end{itemize}

\textbf{専門家レビュー}:動物病院の獣医師2名、ペット関連サービス事業者2名に対し、システムのデモンストレーションを行い、フィードバックを収集した。いずれの専門家からも、実用化の可能性について肯定的な評価を得た。

以上の結果から、提案システムがステイクホルダーのニーズを満たしていることを確認した。

\input{chapters/5_evaluation/5_5_summary}

% 第6章 考察
\chapter{考察}
\section{結果の考察}

\subsection{性能向上の要因}

提案手法が既存手法を上回る性能を達成した要因として、以下の3点が考えられる。

第一に、\textbf{猫特化型のマルチスケール特徴抽出}の効果である。猫の顔・体型・毛色パターンを階層的に統合することで、品種間の微細な差異を捉えることが可能となった。特に、類似品種(例:アメリカンショートヘアとブリティッシュショートヘア)の識別において、既存手法との差が顕著であった。

第二に、\textbf{アテンション機構}の効果である。CBAMの導入により、背景や遮蔽物の影響を軽減し、猫の本質的な特徴に注目することができた。予備調査で課題として挙げられた「部分的に隠れた猫の検出」において、特に効果が見られた。

第三に、\textbf{日本の家庭猫に特化したデータセット}の効果である。既存のベンチマークデータセットには含まれていない日本で人気の品種(スコティッシュフォールド、マンチカン等)の画像を十分に含むことで、国内ユーザーにとって実用的なモデルを構築できた。

\subsection{軽量化と精度のトレードオフ}

本研究では、知識蒸留・量子化・枝刈りを組み合わせることで、精度を維持しながらモデルサイズを削減した。特に、知識蒸留において教師モデル(YOLOv8-large)の中間層特徴も活用したことが、精度維持に寄与したと考えられる。

モデルサイズは7.8MBであり、元のYOLOv8-nanoよりやや大きいが、精度向上(+5.3ポイント)を考慮すると、良好なトレードオフを実現できたと言える。

\subsection{ユーザーテストからの示唆}

ユーザーテストにおいて「使いやすさ」の評価が他の項目より低かった(4.1/5.0)。自由記述の分析から、「結果の説明が不足している」「誤認識時の対処法がわからない」といった意見が見られた。これは、AIシステムの説明可能性(Explainability)に関する課題であり、今後の改善点として認識された。

\input{chapters/6_discussion/6_2_limitations}
\input{chapters/6_discussion/6_3_summary}

% 第7章 結論
\chapter{結論}
\input{chapters/7_conclusion/7_1_conclusion}
\input{chapters/7_conclusion/7_2_future}

% ========== 後付け ==========
% 謝辞
\chapter*{謝辞}
\addcontentsline{toc}{chapter}{謝辞}

本研究を遂行するにあたり、多くの方々にご支援をいただきました。ここに深く感謝の意を表します。

指導教員である鈴木花子教授には、研究の方向性から論文執筆に至るまで、終始丁寧なご指導を賜りました。先生の的確なアドバイスなくして、本研究の完成はあり得ませんでした。心より御礼申し上げます。

副査をお引き受けいただいた田中次郎教授、佐藤三郎准教授には、審査会において貴重なご意見をいただきました。先生方のコメントにより、本研究の質を大きく向上させることができました。

研究室の皆様には、日々の議論やデータ収集において多大なご協力をいただきました。特に、データアノテーション作業に参加いただいた同期の皆様に感謝いたします。

ユーザーインタビューおよびユーザーテストにご協力いただいた猫飼育者の皆様、動物病院・ペット関連事業者の皆様にも御礼申し上げます。実際のニーズをお聞かせいただいたことで、実用的なシステムを設計することができました。

データセット構築にあたり、愛猫の画像をご提供いただいた多くの方々に感謝いたします。皆様のご協力なくして、本研究は成立しませんでした。

最後に、長期にわたる学生生活を支えてくれた家族に感謝します。

2025年3月

山田 太郎


% 参考文献
\begin{thebibliography}{99}

\bibitem{Redmon2016}
Redmon, J., Divvala, S., Girshick, R., \& Farhadi, A. (2016). You Only Look Once: Unified, Real-Time Object Detection. \textit{CVPR 2016}.

\bibitem{YOLOv8}
Jocher, G., et al. (2023). Ultralytics YOLOv8. \textit{GitHub repository}.

\bibitem{Parkhi2012}
Parkhi, O. M., Vedaldi, A., Zisserman, A., \& Jawahar, C. V. (2012). Cats and Dogs. \textit{CVPR 2012}.

\bibitem{Howard2017}
Howard, A. G., et al. (2017). MobileNets: Efficient Convolutional Neural Networks for Mobile Vision Applications. \textit{arXiv:1704.04861}.

\bibitem{Tan2019}
Tan, M., \& Le, Q. V. (2019). EfficientNet: Rethinking Model Scaling for Convolutional Neural Networks. \textit{ICML 2019}.

\bibitem{Hinton2015}
Hinton, G., Vinyals, O., \& Dean, J. (2015). Distilling the Knowledge in a Neural Network. \textit{arXiv:1503.02531}.

\bibitem{Woo2018}
Woo, S., Park, J., Lee, J. Y., \& Kweon, I. S. (2018). CBAM: Convolutional Block Attention Module. \textit{ECCV 2018}.

\bibitem{Schroff2015}
Schroff, F., Kalenichenko, D., \& Philbin, J. (2015). FaceNet: A Unified Embedding for Face Recognition and Clustering. \textit{CVPR 2015}.

\bibitem{He2016}
He, K., Zhang, X., Ren, S., \& Sun, J. (2016). Deep Residual Learning for Image Recognition. \textit{CVPR 2016}.

\bibitem{Lin2017}
Lin, T. Y., et al. (2017). Feature Pyramid Networks for Object Detection. \textit{CVPR 2017}.

\bibitem{PetFood2023}
一般社団法人ペットフード協会 (2023). 令和5年 全国犬猫飼育実態調査.

\bibitem{INCOSE2015}
INCOSE (2015). Systems Engineering Handbook: A Guide for System Life Cycle Processes and Activities, 4th Edition.

\bibitem{Zhang2008}
Zhang, W., et al. (2008). Cat Head Detection - How to Effectively Exploit Shape and Texture Features. \textit{ECCV 2008}.

\bibitem{Liu2012}
Liu, J., et al. (2012). Dog Breed Classification Using Part Localization. \textit{ECCV 2012}.

\bibitem{Kumar2020}
Kumar, S., et al. (2020). Cat Face Recognition Using Deep Learning. \textit{Pattern Recognition Letters}, 139, 52-60.

\end{thebibliography}


% 付録
\appendix
\section*{評価マニュアル概要}

本付録では、NekoNetの性能評価に使用した専門家評価マニュアルを示す。

\subsection*{評価者の選定基準}

\begin{itemize}
\item 画像認識分野での研究経験3年以上
\item 深層学習に関する論文発表実績
\item ペット関連サービスの開発経験(優遇)
\end{itemize}

\subsection*{評価項目}

\begin{enumerate}
\item \textbf{認識精度の主観評価}:1-5のリカートスケール
\item \textbf{処理速度の体感評価}:リアルタイム性の有無
\item \textbf{実用性評価}:実際の製品への導入可能性
\item \textbf{改善提案}:自由記述
\end{enumerate}

\subsection*{評価手順}

各評価者には100枚のテスト画像に対するNekoNetの認識結果を提示し、上記評価項目に基づいて評価を依頼した。評価時間は1人あたり約60分であった。


\end{document}
