% abstract-en.tex
% Abstract(SDM形式)

\thispagestyle{plain}

\begin{center}
{\large\textbf{Abstract}}
\end{center}

\vspace{0.5em}

\noindent
\begin{tabular}{|p{2.2cm}|p{3cm}|p{1.3cm}|p{5.7cm}|}
\hline
Student ID & 12345678 & Name & Taro Yamada \\
\hline
Thesis Title: & \multicolumn{3}{p{10.4cm}|}{Design and Evaluation of a Deep Learning-based Cat Image Recognition System} \\
\hline
\multicolumn{4}{|p{13.1cm}|}{
\vspace{0.2em}
\textbf{(Summary)}

\hspace{1em}With the expansion of the pet industry, demand for automatic cat image recognition technology has been growing. However, existing general-purpose object detection models inadequately address cat-specific morphological features.

\hspace{1em}This study aims to: (1) design and implement ``NekoNet,'' a deep learning system specialized for cat image recognition; (2) realize a lightweight model enabling real-time inference on edge devices; and (3) demonstrate the effectiveness of systematic AI system development applying systems engineering processes.

\hspace{1em}The proposed NekoNet system features a multi-scale feature extraction mechanism, an attention mechanism robust to occlusion and pose variations, and a lightweight architecture. Evaluation experiments using a Japanese domestic cat dataset (15,000 images, 50 breeds) showed that the proposed method achieved mAP@0.5 = 92.6\% in object detection, surpassing YOLOv8 (87.3\%) by 5.3 points. Breed classification achieved Top-1 Accuracy of 87.3\%, and individual identification achieved Rank-1 Accuracy of 93.4\%. A model size of 7.8MB and 48 FPS in mobile environments were realized.

\hspace{1em}This study contributes to establishing a new benchmark for cat image recognition and demonstrating an AI system development process using systems engineering.
\vspace{0.2em}
} \\
\hline
\multicolumn{4}{|p{13.1cm}|}{
\textbf{Keywords (5 words)}

Deep Learning, Object Detection, Cat Image Recognition, Model Compression, Systems Engineering
\vspace{0.2em}
} \\
\hline
\end{tabular}
