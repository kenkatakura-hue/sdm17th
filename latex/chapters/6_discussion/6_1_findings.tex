\section{結果の考察}

\subsection{性能向上の要因}

提案手法が既存手法を上回る性能を達成した要因として、以下の3点が考えられる。

第一に、\textbf{猫特化型のマルチスケール特徴抽出}の効果である。猫の顔・体型・毛色パターンを階層的に統合することで、品種間の微細な差異を捉えることが可能となった。特に、類似品種(例:アメリカンショートヘアとブリティッシュショートヘア)の識別において、既存手法との差が顕著であった。

第二に、\textbf{アテンション機構}の効果である。CBAMの導入により、背景や遮蔽物の影響を軽減し、猫の本質的な特徴に注目することができた。予備調査で課題として挙げられた「部分的に隠れた猫の検出」において、特に効果が見られた。

第三に、\textbf{日本の家庭猫に特化したデータセット}の効果である。既存のベンチマークデータセットには含まれていない日本で人気の品種(スコティッシュフォールド、マンチカン等)の画像を十分に含むことで、国内ユーザーにとって実用的なモデルを構築できた。

\subsection{軽量化と精度のトレードオフ}

本研究では、知識蒸留・量子化・枝刈りを組み合わせることで、精度を維持しながらモデルサイズを削減した。特に、知識蒸留において教師モデル(YOLOv8-large)の中間層特徴も活用したことが、精度維持に寄与したと考えられる。

モデルサイズは7.8MBであり、元のYOLOv8-nanoよりやや大きいが、精度向上(+5.3ポイント)を考慮すると、良好なトレードオフを実現できたと言える。

\subsection{ユーザーテストからの示唆}

ユーザーテストにおいて「使いやすさ」の評価が他の項目より低かった(4.1/5.0)。自由記述の分析から、「結果の説明が不足している」「誤認識時の対処法がわからない」といった意見が見られた。これは、AIシステムの説明可能性(Explainability)に関する課題であり、今後の改善点として認識された。
