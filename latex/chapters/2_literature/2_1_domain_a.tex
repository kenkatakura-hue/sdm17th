\section{深層学習に基づく物体検出}

物体検出は、画像中の物体の位置(バウンディングボックス)とクラスを同時に推定するタスクである。深層学習の発展以前は、HOG特徴量とSVMを組み合わせた手法が主流であったが、2012年のAlexNetの登場以降、畳み込みニューラルネットワーク(CNN)に基づく手法が急速に発展した。

物体検出の深層学習手法は、大きく2段階検出器と1段階検出器に分類される。2段階検出器の代表例であるR-CNNファミリーは、領域提案ネットワーク(RPN)で候補領域を生成した後、各領域を分類する。Fast R-CNN、Faster R-CNNと改良が重ねられ、高精度な検出が可能となった。一方、YOLOやSSDに代表される1段階検出器は、領域提案を省略して直接検出を行うため、リアルタイム処理に適している。

近年では、Transformerアーキテクチャを物体検出に応用したDETR(DEtection TRansformer)が登場し、エンドツーエンドの学習を実現している。また、YOLOv8やRT-DETRなど、精度と速度のバランスに優れた最新手法も提案されている。

しかしながら、これらの汎用手法は特定のドメイン(本研究では猫画像)に最適化されておらず、ドメイン固有の特徴を活かした手法の開発が求められている。
