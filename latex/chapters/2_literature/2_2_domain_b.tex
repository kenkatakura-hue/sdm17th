\section{猫画像認識に関する先行研究}

猫画像認識に関する研究は、主に以下の3つのタスクに分類される。

\subsection{猫検出}

画像中から猫の存在を検出するタスクである。Parkhi et al. (2012) は Oxford-IIIT Pet Datasetを公開し、37品種の犬猫画像を用いたベンチマークを確立した。Zhang et al. (2008) は、猫の顔検出に特化したカスケード分類器を提案し、毛色や模様のバリエーションに対応した。近年では、汎用物体検出モデルの転移学習により、高い検出精度が達成されている。

\subsection{猫種分類}

検出された猫を品種ごとに分類するタスクである。細粒度画像認識(Fine-Grained Visual Categorization)の一種として研究されており、品種間の微細な差異を捉えることが課題となる。Liu et al. (2012) は、局所特徴量と大域特徴量を組み合わせた手法を提案した。また、アテンション機構を用いて識別に重要な領域を自動的に特定する研究も行われている。

\subsection{個体識別}

同一品種内の個体を識別するタスクである。ペットの迷子捜索や健康管理において重要なアプリケーションである。顔認証技術を応用したCat Face Recognitionや、毛色パターンを用いた手法が提案されている。Kumar et al. (2020) は、Siamese Networkを用いた猫の個体識別システムを開発し、95\%以上の識別精度を報告している。

これらの先行研究は個別のタスクに焦点を当てているが、検出から個体識別までを統合的に扱うシステムの研究は限定的である。
