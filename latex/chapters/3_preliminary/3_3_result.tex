\section{調査結果}

\subsection{既存システム分析の結果}

10アプリケーションの分析から、以下の知見が得られた。

\textbf{機能面}:品種識別機能を持つアプリは7件、個体識別機能を持つアプリは2件であった。複数頭の同時検出に対応しているアプリは4件にとどまった。

\textbf{精度面}:自前の100枚テストでは、最高精度のアプリでも品種識別の正解率は73\%であった。特に、混血猫や希少品種での誤認識が多く見られた。また、暗い環境や猫が動いている場面での認識精度が著しく低下した。

\textbf{ユーザビリティ}:ユーザーレビューの分析から、「認識に時間がかかる」「オフラインで使えない」「日本の猫種に対応していない」といった不満が頻出していた。

\subsection{ユーザーインタビューの結果}

インタビューの質的分析から、以下の主要なニーズが抽出された。

\begin{enumerate}
    \item \textbf{迷子対策}:猫が迷子になった際に、写真から個体を特定できる機能への要望(飼育者12名が言及)
    \item \textbf{健康管理}:体型や毛並みの変化を追跡し、健康状態を把握したいというニーズ(飼育者8名、獣医師3名が言及)
    \item \textbf{多頭識別}:複数の猫を飼育している場合に、個体ごとの記録を管理したいというニーズ(飼育者6名が言及)
    \item \textbf{リアルタイム性}:撮影と同時に結果が得られることへの期待(全グループで言及)
\end{enumerate}

\subsection{プロトタイプテストの結果}

YOLOv8ベースのプロトタイプは、500枚のテスト画像に対して以下の結果を示した。

\begin{itemize}
    \item 検出精度(mAP@0.5):78.3\%
    \item 品種分類精度(Top-1 Accuracy):65.2\%
    \item 処理速度:42 FPS(GPU環境)、8 FPS(CPU環境)
\end{itemize}

失敗パターンの分析から、以下の課題が明らかになった:(1)類似品種間の混同、(2)部分的に隠れた猫の検出失敗、(3)丸まった姿勢での認識精度低下。
