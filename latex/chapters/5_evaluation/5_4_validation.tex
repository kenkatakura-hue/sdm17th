\section{検証と妥当性確認}

\subsection{システム要求の検証結果}

表\ref{tab:verification}に、各システム要求の検証結果をまとめる。

\begin{table}[htbp]
\centering
\caption{システム要求の検証結果}
\label{tab:verification}
\begin{tabular}{llcc}
\toprule
ID & 要求内容 & 目標値 & 達成値 \\
\midrule
SyR1 & 検出精度(mAP@0.5) & $\geq$ 92\% & 92.6\% ✓ \\
SyR2 & 分類精度(Top-1 Acc.) & $\geq$ 85\% & 87.3\% ✓ \\
SyR3 & 識別精度(Rank-1 Acc.) & $\geq$ 92\% & 93.4\% ✓ \\
SyR4 & 処理速度(Mobile) & $\geq$ 30 FPS & 48 FPS ✓ \\
SyR5 & モデルサイズ & $\leq$ 10 MB & 7.8 MB ✓ \\
SyR6 & オンデバイス推論 & 対応 & 対応 ✓ \\
\bottomrule
\end{tabular}
\end{table}

すべてのシステム要求が達成されたことを確認した。

\subsection{妥当性確認}

システムが実際のユーザーニーズを満たしているかを確認するため、以下の妥当性確認を実施した。

\textbf{ユーザーテスト}:猫飼育者10名に対し、プロトタイプアプリケーションを2週間使用してもらい、満足度調査を実施した。5段階評価(5が最高)で、以下の結果を得た。

\begin{itemize}
    \item 認識精度への満足度:4.3
    \item 処理速度への満足度:4.6
    \item 使いやすさへの満足度:4.1
    \item 総合満足度:4.2
\end{itemize}

\textbf{専門家レビュー}:動物病院の獣医師2名、ペット関連サービス事業者2名に対し、システムのデモンストレーションを行い、フィードバックを収集した。いずれの専門家からも、実用化の可能性について肯定的な評価を得た。

以上の結果から、提案システムがステイクホルダーのニーズを満たしていることを確認した。
