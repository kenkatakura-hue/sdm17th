\section{本論文の構成}

本論文は全7章から構成される。

\textbf{第1章}では、本研究の背景、目的、新規性、および論文の構成を述べた。

\textbf{第2章}では、先行研究のレビューを行う。深層学習に基づく物体検出手法、猫画像認識の既存研究、および軽量化技術について整理し、本研究の位置づけを明確にする。

\textbf{第3章}では、予備調査の結果を報告する。既存システムの分析、ユーザーインタビュー、およびプロトタイプテストを通じて得られた知見を整理し、システム設計への示唆を導出する。

\textbf{第4章}では、提案手法であるNekoNetの設計を詳述する。システムズエンジニアリングに基づく要求定義、アーキテクチャ設計、およびアルゴリズムの詳細を説明する。

\textbf{第5章}では、評価実験について述べる。実験の目的、方法、結果を報告し、提案手法の有効性を検証する。

\textbf{第6章}では、実験結果に基づく考察を行う。得られた知見の解釈、および本研究の限界について議論する。

\textbf{第7章}では、本研究の結論と今後の展望を述べる。
