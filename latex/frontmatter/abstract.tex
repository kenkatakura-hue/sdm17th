% abstract.tex
% 論文要旨(SDM形式)

\thispagestyle{plain}

\begin{center}
{\large\textbf{論 文 要 旨}}
\end{center}

\vspace{1em}

\noindent
\begin{tabular}{|p{2cm}|p{3.2cm}|p{1.3cm}|p{5.7cm}|}
\hline
学籍番号 & 12345678 & 氏 名 & 山田太郎 \\
\hline
論文題目: & \multicolumn{3}{p{10.6cm}|}{深層学習を用いた猫画像認識システムの設計と評価} \\
\hline
\multicolumn{4}{|p{13.1cm}|}{
\vspace{0.3em}
\textbf{(内容の要旨)}

\hspace{1em}近年、ペット関連産業の拡大に伴い、猫画像の自動認識技術への需要が高まっている。しかし、既存の汎用物体検出モデルは猫特有の形態学的特徴への対応が不十分であり、特に多頭飼育環境での個体識別や、猫種の細粒度分類において精度の限界がある。

\hspace{1em}本研究の目的は、(1) 猫画像認識に特化した深層学習システム「NekoNet」を設計・実装し、既存手法を上回る検出・分類精度を達成すること、(2) スマートフォン等のエッジデバイスでリアルタイム推論が可能な軽量モデルを実現すること、(3) システムズエンジニアリングプロセスを適用した体系的なAIシステム開発の有効性を実証すること、の3点である。

\hspace{1em}提案システムNekoNetは、猫の顔・体型・毛色パターンを統合的に解析するマルチスケール特徴抽出機構、遮蔽や姿勢変化に頑健なアテンション機構、およびエッジデバイスでの実行を可能にする軽量アーキテクチャを特徴とする。独自に構築した日本の家庭猫データセット(15,000枚、50品種)を用いた評価実験の結果、提案手法は物体検出においてmAP@0.5 = 92.6\%を達成し、既存手法YOLOv8(87.3\%)を5.3ポイント上回った。品種分類ではTop-1 Accuracy 87.3\%、個体識別ではRank-1 Accuracy 93.4\%を達成した。また、知識蒸留・量子化・枝刈りの組み合わせにより、モデルサイズ7.8MB、モバイル環境で48FPSの実行性能を実現した。

\hspace{1em}本研究は、猫画像認識における新たなベンチマークの確立、実用的な軽量モデルの設計指針の提示、およびシステムズエンジニアリングを用いたAIシステム開発プロセスの実証に貢献する。
\vspace{0.3em}
} \\
\hline
\multicolumn{4}{|p{13.1cm}|}{
\textbf{キーワード(5語)}

深層学習、物体検出、猫画像認識、軽量化、システムズエンジニアリング
\vspace{0.2em}
} \\
\hline
\end{tabular}