% abstract.tex
% 論文要旨(SDM形式)

\thispagestyle{plain}

\begin{center}
{\large\textbf{論 文 要 旨}}
\end{center}

\vspace{1em}

\noindent
\begin{tabular}{|p{2cm}|p{3.2cm}|p{1.3cm}|p{5.7cm}|}
\hline
学籍番号 & 12345678 & 氏 名 & 山田太郎 \\
\hline
論文題目: & \multicolumn{3}{p{10.6cm}|}{猫画像認識AIの効率化手法に関する研究 --- NekoNetアーキテクチャの提案と評価 ---} \\
\hline
\multicolumn{4}{|p{13.1cm}|}{
\vspace{0.3em}
\textbf{(内容の要旨)}

\hspace{1em}深層学習を用いた画像認識技術は急速に発展しているが、猫画像の認識においては毛並みパターンの多様性や姿勢変化への対応が課題となっている。既存の汎用画像認識モデルでは、これらの猫特有の課題に対して十分な対応ができておらず、猫の個体識別においては80\%程度の精度に留まることが報告されている。

\hspace{1em}本研究の目的は、(1) 猫画像認識に関連する既存手法を体系的にレビューし、現状の課題と研究ギャップを明らかにすること、(2) 猫特有の形態学的特徴を活用した新規ニューラルネットワークアーキテクチャ「NekoNet」をISO 15288に基づくシステムズエンジニアリングアプローチを用いて設計・実装・評価すること、の2点である。

\hspace{1em}体系的文献レビュー(SLR)により、417件の論文から猫認識に関連する42手法を抽出・分類した。分析の結果、既存手法の大半が猫特有の形態学的特徴を活用しておらず、精度と速度のトレードオフが大きいことを発見した。この課題に対応するため、猫の耳・目・尾の形態学的特徴に着目したNeko Attention Moduleを設計し、MobileNetV3をベースとした軽量アーキテクチャNekoNetを実装した。評価実験の結果、提案手法はmAP@0.5で92.6\%の認識精度と48FPSの処理速度を達成し、従来手法(YOLOv8-nano、MobileNet-SSD、EfficientDet-D0)を有意に上回る性能を示した。

\hspace{1em}本研究は、猫画像認識における形態学的特徴の体系化と、それを活用した効率的な認識手法の確立に貢献する。
\vspace{0.3em}
} \\
\hline
\multicolumn{4}{|p{13.1cm}|}{
\textbf{キーワード(5語)}

深層学習、画像認識、畳み込みニューラルネットワーク、アテンション機構、NekoNet
\vspace{0.2em}
} \\
\hline
\end{tabular}