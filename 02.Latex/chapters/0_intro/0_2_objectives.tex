本研究の目的は以下の3点である。

第一に、猫画像認識に関連する既存手法を体系的にレビューし、現状の課題と研究ギャップを明らかにすることである。体系的文献レビュー(Systematic Literature Review: SLR)の手法を用いて、客観的かつ再現可能な形で既存研究を整理する。

第二に、猫特有の形態学的特徴を活用した新規ニューラルネットワークアーキテクチャ「NekoNet」を設計・実装することである。設計にあたっては、ISO 15288に基づくシステムズエンジニアリングアプローチを採用し、要求から実装までのトレーサビリティを確保する。

第三に、提案手法の有効性を定量的に評価することである。複数のベンチマークデータセットを用いた実験により、認識精度と処理速度の両面から従来手法との比較評価を行う。
