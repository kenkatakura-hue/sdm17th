NekoNetのアーキテクチャ設計について述べる。

\subsection{設計方針}

SLRで得られた知見に基づき、以下の設計方針を採用する。

\begin{enumerate}
\item 猫の形態学的特徴(耳・目・尾)に着目したアテンション機構
\item 軽量化のためのDepthwise Separable Convolutionの採用
\item マルチスケール特徴抽出によるサイズ不変性の確保
\end{enumerate}

\subsection{全体アーキテクチャ}

NekoNetは以下の3つの主要コンポーネントから構成される(図\ref{fig:arch})。

\begin{figure}[htbp]
\centering
\fbox{\parbox{0.8\textwidth}{\centering NekoNetアーキテクチャ図\\Backbone → Neko Attention → Detection Head}}
\caption{NekoNetアーキテクチャ}
\label{fig:arch}
\end{figure}

\begin{enumerate}
\item \textbf{Backbone}:MobileNetV3をベースとした特徴抽出器
\item \textbf{Neko Attention Module}:猫特有部位に着目するアテンション機構
\item \textbf{Detection Head}:位置・クラス予測を行う出力層
\end{enumerate}
