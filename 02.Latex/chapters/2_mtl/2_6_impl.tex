実装の詳細について述べる。

\subsection{開発環境}

\begin{itemize}
\item フレームワーク:PyTorch 2.0
\item 学習環境:NVIDIA A100 GPU × 4
\item 推論環境:Raspberry Pi 4(評価用エッジデバイス)
\end{itemize}

\subsection{データセット}

学習には以下のデータセットを使用した。

\begin{table}[htbp]
\centering
\caption{使用データセット}
\begin{tabular}{lrr}
\toprule
データセット & 画像数 & 猫インスタンス数 \\
\midrule
Oxford-IIIT Pet & 7,349 & 7,349 \\
COCO (cat) & 12,438 & 15,234 \\
自作データセット & 5,000 & 6,812 \\
\midrule
合計 & 24,787 & 29,395 \\
\bottomrule
\end{tabular}
\end{table}

\subsection{学習設定}

バッチサイズ32、初期学習率0.001、エポック数100で学習を行った。最適化にはAdamWを使用し、コサインアニーリングによる学習率スケジューリングを適用した。
