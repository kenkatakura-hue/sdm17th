本章では、ISO 15288に基づくシステムズエンジニアリングアプローチを用いて、NekoNetの設計・実装・評価を行った。

主な成果は以下の通りである。

\begin{enumerate}
\item ビジネス分析から実装・評価まで、トレーサビリティを確保した開発を実施
\item 猫特有の形態学的特徴を活用したNeko Attention Moduleを設計
\item mAP@0.5で92.6\%の認識精度を達成(従来手法比+7.3〜14.2ポイント)
\item 48FPSのリアルタイム処理を実現
\item モデルサイズ7.8MBでエッジデバイス展開に適した軽量化を達成
\end{enumerate}

これらの結果は、全てのシステム要求(SyRS)を満たしており、提案手法の有効性が確認された。
