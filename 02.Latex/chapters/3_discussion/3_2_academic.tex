本研究の学術的貢献は以下の3点である。

\subsection{猫画像認識分野の体系化}

本研究で実施したSLRは、猫画像認識分野における初の包括的な文献レビューである。提案した分類体系(CNN系・物体検出系・セグメンテーション系・ハイブリッド系)は、今後の研究の位置づけに有用な枠組みを提供する。

\subsection{形態学的特徴の活用手法}

Neko Attention Moduleは、ドメイン知識をニューラルネットワークのアテンション機構に統合する新たなアプローチを示した。この手法は猫以外の動物認識にも応用可能であり、汎用性を有する。

\subsection{設計科学研究の方法論的貢献}

ISO 15288とDSRフレームワークを統合した研究アプローチは、AI/ML分野における設計科学研究の一つの雛形を提示するものである。
