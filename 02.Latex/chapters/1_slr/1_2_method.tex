本研究では、Kitchenham \& Chartersのガイドラインに基づくSLRプロトコルを採用した。

\subsection{検索戦略}

検索対象データベースとして、IEEE Xplore、ACM Digital Library、Scopus、Web of Scienceの4つを選定した。検索クエリは以下の通りである。

\begin{verbatim}
("cat" OR "feline") AND ("image recognition" OR 
"object detection" OR "deep learning" OR "CNN")
\end{verbatim}

検索期間は2015年1月から2024年12月までとした。

\subsection{選定基準}

包含基準(Inclusion Criteria)および除外基準(Exclusion Criteria)を表\ref{tab:criteria}に示す。

\begin{table}[htbp]
\centering
\caption{文献選定基準}
\label{tab:criteria}
\begin{tabular}{ll}
\toprule
種別 & 基準 \\
\midrule
IC1 & 猫画像認識を主題とする査読付き論文 \\
IC2 & 定量的な評価結果を報告している \\
EC1 & 会議予稿・ポスター発表のみ \\
EC2 & 英語以外の言語で記述 \\
\bottomrule
\end{tabular}
\end{table}

\subsection{スクリーニングプロセス}

2名の研究者が独立してスクリーニングを実施し、不一致については協議により解決した。Cohen's Kappaは0.87であり、高い一致率を示した。
