本章の目的は、猫画像認識に関連する既存手法を体系的にレビューし、研究ギャップと今後の方向性を明らかにすることである。

具体的には、以下の研究課題(Research Questions: RQ)を設定した。

\begin{description}
\item[RQ1] 猫画像認識において、どのような手法が提案されているか?

本RQでは、猫画像認識に適用されてきた手法の全体像を把握することを目指す。CNN、物体検出、セグメンテーション等の技術的アプローチを網羅的に収集し、分類体系を構築する。これにより、研究分野の成熟度と多様性を評価する基盤を得る。

\item[RQ2] 各手法の性能(精度・速度)はどの程度か?

本RQでは、既存手法の定量的な性能比較を行う。認識精度(mAP、Accuracy等)と処理速度(FPS、推論時間)の両面から各手法を評価し、精度と速度のトレードオフ関係を明らかにする。この分析により、実用化に向けた技術的課題を特定する。

\item[RQ3] 既存手法の課題と限界は何か?

本RQでは、既存研究において未解決の問題や改善の余地がある領域を特定する。特に、猫特有の形態学的特徴(毛並みの多様性、姿勢変化、品種間差異)への対応状況を分析し、新規アーキテクチャ開発に向けた研究ギャップを明確化する。
\end{description}

これらの研究課題に対して、体系的文献レビュー(SLR)の手法を用いて客観的に回答を導出する。SLRを採用することで、研究者の主観的バイアスを排除し、再現可能かつ網羅的な文献調査を実現する。