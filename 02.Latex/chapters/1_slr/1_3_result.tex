初期検索により250件の文献が同定された。重複除去後185件となり、タイトル・アブストラクトスクリーニングで46件に絞られた。最終的にフルテキストレビューを経て23件が分析対象として選定された(図\ref{fig:prisma})。

\begin{figure}[htbp]
\centering
\fbox{\parbox{0.8\textwidth}{\centering PRISMAフローダイアグラム\\(検索250件 → 最終23件)}}
\caption{文献選定プロセス}
\label{fig:prisma}
\end{figure}

\subsection{手法の分類}

選定された23件の文献で提案された手法は、以下の4カテゴリに分類された。

\begin{enumerate}
\item CNN系(10件):ResNet、VGG等の汎用アーキテクチャを適用
\item 物体検出系(5件):YOLO、Faster R-CNN等を猫検出に適用
\item セグメンテーション系(6件):猫の輪郭抽出に特化
\item ハイブリッド系(2件):複数手法の組み合わせ
\end{enumerate}

\subsection{性能比較}

報告された認識精度は62.3\%から94.7\%の範囲であり、平均は81.2\%であった。処理速度は1フレームあたり12msから340msと大きなばらつきが見られた。
