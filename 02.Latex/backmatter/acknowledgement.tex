本研究を遂行するにあたり、多くの方々にご支援をいただきました。

指導教員の福沢諭吉先生には、研究の構想段階から論文執筆に至るまで、終始懇切丁寧なご指導を賜りました。先生の深い洞察と温かい励ましがなければ、本研究を完成させることはできませんでした。心より感謝申し上げます。

副査の田中一郎教授、佐藤次郎准教授には、中間発表や予備審査において貴重なご意見をいただきました。深く御礼申し上げます。

研究室の皆様には、日々の議論や実験のサポートなど、多大なご協力をいただきました。特に、データセット作成にご協力いただいた後輩の皆さんに感謝いたします。

また、愛猫のミケとタマには、研究のインスピレーションを与えてくれたことに感謝します。彼らの愛らしい姿が、本研究の原動力となりました。

最後に、長期間にわたる研究生活を支えてくれた家族に、心からの感謝を捧げます。

\vspace{1cm}
\begin{flushright}
2026年3月\\
慶應 太郎
\end{flushright}
